\documentclass{bioinfo}
\copyrightyear{2014}
\pubyear{2014}

\usepackage{natbib}
\bibliographystyle{apalike}

\begin{document}
\firstpage{1}

\title[short Title]{DNAism: Exploring genomic datasets on the web with Horizon Charts. }
%\author[Sample \textit{et~al}]{David Rio Deiros\,$^{1,*}$, XXXXXXXXXX\,$^{2}$ and Jeff Rogers\,$^2$\footnote{to whom correspondence should be addressed}}
\author[Sample \textit{et~al}]{David Rio Deiros\,$^{1}$ and Jeff Rogers\,$^1$\footnote{to whom correspondence should be addressed}}
\address{$^{1}$Department of Genomic, Baylor College of Medicine. }
%\address{$^{1}$Department of Genomic, Baylor College of Medicine. \\
%$^{2}$Department of XXXXXXXX, Address XXXX etc.}

\history{Received on XXXXX; revised on XXXXX; accepted on XXXXX}

\editor{Associate Editor: XXXXXXX}

\maketitle

\begin{abstract}

\section{Summary:}
Computational biologist  daily face the need of exploring massive amounts of
genomic datasets. New visualization techniques help researchers navigate and
understand these big data sets. Horizon Charts are a relatively new
visualization method that, under the right conditions~\citep{2009-horizon}, maximize data density
without losing graphical perception. This visualization technique has been
successfully applied to understand multi-metric time series data. We have
adapted an existing Javascript library (Cubism) that implements Horizon Charts
for the time series domain to make it work with genomic datasets. We call this
new library DNAism. For big data sets we provide a backend server implemented
in python that relies on tabix to speed up the acquisition of the data points
necessary for the visualizations. Finally, we have built a small web
application to demonstrate the usefulness and effectiveness of our library.

\section{Availability and implementation:}
http://github.com/drio/dnaism

\section{Contact:} \href{deiros@bcm.edu}{deiros@bcm.edu}
\end{abstract}

\section{Introduction}

Sharing and communicating about large and intricate datasets produced by
Next-Gen sequencing can be a challenging task. Visual channels are an effective
way to explore data. However, the accelerating increase in data quantity is
pushing the limits of current approaches in representing these datasets
visually, without sacrificing accuracy or graphical perception.  Data volume is
increasing vertically as throughput increases, and horizontally as studies
involving large numbers of subjects become more feasible.  Thus, more effective
visualization techniques are needed to understand the most challenging Next-Gen
Sequencing datasets.

Horizon Charts {1} have proved to be a very effective visualization approach
when working with multi-metric time-series encoded data. BED files are the gold
standard for capturing genomic metrics in the Next-Gen sequencing domain. In
time series, metrics are monitored over time, however, BED files use genomic
coordinates instead. We have adapted a time series Javascript library to the
genomic domain. Our new library is called DNAism.

\begin{figure*}
\centerline{\includegraphics{figure.pdf}}
\caption{
Horizon charts emerge from applying a set of changes to traditional line
charts. We start by coloring the underlying area of the line chart, using
different colors for positive and negative values (A). Then, we divide the
graph in bands and apply a gradient of color that increases along with the
quantitative value of the variable we are representing (B). In a later step,
the negative values are flipped over the baseline (C), effectively reducing the
vertical space by two fold. In a final step, bands are collapsed making all of
them start at the baseline and providing another level of space reduction (D).
Visualization of read depth across 24 Whole Genome Sequencing Rhesus samples
(30x coverage) for genomic region Chr17:1.1M-1.2M (E). The high levels of data
density facilitates the perception of patterns across the whole dataset.
}\label{fig:01}
\end{figure*}




\section{Implementation}

Cubism {1} is a Javascript library (based on d3 {6}) that implements horizon
charts using the web platform in the context of time series data. In genomic
datasets, the variable under study is associated with chromosomal coordinates
instead of timestamps. We have adapted the library to support genomic datasets
without modifying the powerful and elegant original interface to make DNAism a
flexible and effective tool to explore multi-sample genomic datasets with
horizon charts. Selecting this platform yields another benefit for DNAism
users: They can build richer and more interactive visualizations by leveraging
the power of the web ecosystem and their ubiquitous technologies (CSS, HTML and
Javascript).

The modifications made to the original library are dispersed across all the
different components since time values are associated with all the data points.
We have modified the majority of the components to use genomic coordinates
instead. In addition, we wrote two major source components to load data from
genomic formats (BED files). The source component encapsulates the details on
how to retrieve data and answer queries from the other library pieces. Our most
simple source (�bedfile.js�) loads in memory all the data from the input file
and returns the necessary data points for the current context when queried. The
context is the principal component of the library and where we define the
region of the genome we want to explore.

Loading all the data from the file (or files) will be inefficient for bigger
datasets, especially when exploring multi-sample data. To overcome this, the
user may choose to precompute the necessary data points for a particular region
to minimize the amount of data points loaded in by the browser. We offer
another alternative via �bedserver.js�. With this source, the browser sends
queries to a dedicated server that implements a RESTful API. This backend
relies on indexed data{4:tabix} to speed up random access to the data and only
returns back to the browser the necessary data points for the visualization.
Along with the library we provide a server (distributed as a python package:
bedserver{5}) that implements this concept.

The source code has a clean and decoupled interface which facilitates the
extension of the library. DNAism is data agnostic and users can create new
sources to accommodate their backend needs.

\section{Results}

We have exposed to the genomic community a powerful visualization technique
previously used in the time series data domain. Our library keeps the effective
and elegant interface of the original, allowing users to leverage its power in
the context of the web platform.

By providing a library, we have the flexibility on how to use these resources,
either by building full applications or using it within existing ones.

The companion lightweight server will facilitate the exploration of large
genomic datasets without affecting user experience by using indexed data sets.
Alternatively, users can create their own data sources to capture the
peculiarities of their data containers.

To expose the power of DNAism, we accompany with DNAism a web application
example that uses the library and the bedserver backend that facilitates
exploring and comparing genomic metrics from different projects and samples.

\section{Discussion}

XXXXXXXX

\section*{Acknowledgement}
Text text

\paragraph{Funding\textcolon} Text Text Text Text Text Text  Text Text.

\bibliography{document}

\end{document}
