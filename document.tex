\documentclass{bioinfo}
\copyrightyear{2014}
\pubyear{2014}

\usepackage{natbib}
%\bibliographystyle{plain}
\bibliographystyle{apalike}

\begin{document}
\firstpage{1}

\title[short Title]{DNAism: Exploring genomic datasets on the web with Horizon Charts. }
%\author[Sample \textit{et~al}]{David Rio Deiros\,$^{1,*}$, XXXXXXXXXX\,$^{2}$ and Jeff Rogers\,$^2$\footnote{to whom correspondence should be addressed}}
\author[Sample \textit{et~al}]{David Rio Deiros\,$^{1}$, Richard Gibbs,$^{1}$ and Jeff Rogers\,$^1$\footnote{to whom correspondence should be addressed}}
\address{$^{1}$Department of Genomics, Baylor College of Medicine. }
%\address{$^{1}$Department of Genomic, Baylor College of Medicine. \\
%$^{2}$Department of XXXXXXXX, Address XXXX etc.}

\history{Received on September 18,2013 ; revised on September 25,2013; accepted on September 26,2013}

\editor{Associate Editor: Name Here.}

\maketitle

\begin{abstract}

\section{Summary:}

Computational biologists daily face the need of exploring massive amounts of
genomic data. New visualization techniques help researchers navigate and
understand these big data. Horizon Charts are a relatively new visualization
method that, under the right circumstances~\citep{2009-horizon}, maximizes data
density without losing graphical perception. This visualization technique has
been successfully applied to understand multi-metric time series data. We have
adapted an existing Javascript library
(Cubism\footnote{http://github.com/square/cubism}) that implements Horizon
Charts for the time series domain so that it works efficiently and easily with
genomic datasets. We call this new library DNAism. For big data sets we provide
a backend server implemented in Python that relies on tabix\citep{tabix-li} to
efficiently retrieve the data points necessary for the visualizations. Finally,
we have built a small web application to demonstrate the usefulness and
effectiveness of our library.

\section{Availability and implementation:}
Source code, documentation and examples can ben found at: http://github.com/drio/dnaism.

\section{Contact:} \href{deiros@bcm.edu}{deiros@bcm.edu}
\end{abstract}

\section{Introduction}

Sharing and communicating about large and intricate datasets produced by
Next-Gen sequencing can be a challenging task. Visual channels are an effective
way to explore data. However, the accelerating increase in data quantity is
pushing the limits of current approaches to representing these datasets
visually without sacrificing accuracy or graphical perception.  Data volume is
increasing vertically as throughput per sample increases, and horizontally as
studies involving large numbers of subjects become more feasible.  Thus, more
effective visualization techniques are needed to understand the most
challenging Next-Gen sequencing datasets.

Horizon Charts~\citep{time-in-the-horizon} have proved to be an effective
visualization approach when working with multi-metric time-series encoded data.
On the other hand, BED\footnote{http://genome.ucsc.edu/FAQ/FAQformat.html}
files are the gold standard for capturing genomic metrics in the Next-Gen
sequencing domain. In time series, metrics are monitored over time, however,
BED files use genomic coordinates. We have adapted a time series javascript
library to the genomic domain. We call our new library DNAism.


\begin{figure*}
\centerline{\includegraphics{figure.pdf}}
\caption{
Horizon Charts emerge from applying a set of changes to
traditional line charts. We start by coloring the underlying area of the line
chart, using different colors for positive and negative values \textbf{(A)}. Then, we
divide the graph in bands and apply a gradient of color that increases along
with the quantitative value of the variable we are observing \textbf{(B)}. In a later
step, the negative values are flipped over the baseline \textbf{(C)}, effectively
reducing the vertical space by two fold. In a final step, bands are collapsed
making all of them start at the baseline and providing another level of space
reduction \textbf{(D)}.  Visualization of read depth across whole genome sequences from
24 rhesus macaque samples (30x coverage) for genomic region Chr17:1.1M-1.2M
\textbf{(E)}. In this snapshot we can appreciate the strengths of the library and
visualization technique: First, extraordinary patterns are easily observed.
Second, comparing the metric values across all the sample is simple and
effective (There are regions consistently underrepresented across all the
samples and 32510 has low coverage across the genomic region). We are
effectively conveying information from 500k data points (notice there is a
downsampling process to map input values to the available area for the
visualization).
}\label{fig:01}
\end{figure*}



\section{Implementation}

Cubism is a Javascript library (based on D3~\citep{2011-d3}) that implements
Horizon Charts using the web platform in the context of time series data. In
genomic datasets, the variable under study is associated with chromosomal
coordinates instead of timestamps. We have adapted the library to support
genome coordinate data without modifying the powerful and elegant original
interface.  This makes DNAism a flexible and effective tool to explore
multi-sample genomic datasets with Horizon Charts. Selecting this platform
yields another benefit for DNAism users: They can build richer and more
interactive visualizations by leveraging the power of the web ecosystem and its
ubiquitous technologies (CSS, HTML and Javascript).

Functions within the different components are the main entry points to the
library. The main component is context, where we define the region of the
genome we want to explore. We also specify, in pixels, how much vertical space
we have available for the visualization. Visual elements for the horizontal
axis are encapsulated in the axis component. A vertical line can be created to
inspect metric values, across all samples, for specific locations. This is
controlled via the rule component. Access to the data is performed via sources
which contain the logic to pull and map the necessary data for the
visualization. The source component specifies a software contract that all the
sources have to follow in order to integrate correctly. Our library provides
two sources: bedfile and bedserver. Once the sources are created we can use the
metric component to instantiate metrics pointing to specific samples. Finally,
the horizon component encapsulates all the necessary logic to create the visual
elements necessary for the visualization.

The modifications made to the original library are dispersed across all the
different components since time values are associated with all the data points.
We have modified the majority of the components to use genomic coordinates
instead. In addition, we wrote two major source components to load data from
genomic formats (BED files). As explained earlier, the source component
encapsulates the details on how to retrieve data and answer queries from the
other library pieces. Our most simple source (�bedfile.js�) loads in memory all
the data from the input file and returns the necessary data points for the
current context when queried. The context is the principal component of the
library and where we define the region of the genome we want to explore.

Loading all the data from the file (or files) will be inefficient for bigger
datasets, especially when exploring multi-sample data. To overcome this, the
user may choose to precompute the necessary data points for a particular region
to minimize the number of data points loaded in by the browser. We offer
another alternative via �bedserver.js�. With this source, the browser sends
queries to a dedicated server that implements a RESTful API. This backend
relies on indexed data~\citep{tabix-li} to speed up random access to the data
and only returns back to the browser the necessary data points for the
visualization.  Along with the library we provide a server (distributed as a
Python package: bedserver\footnote{https://github.com/drio/bedserver}) that
implements this concept.

The source code has a clean and decoupled interface which facilitates the
extension of the library. DNAism is data agnostic and users can create new
sources to capture their specific backend requirements.

\section{Results}

We introduce the genomics community to a powerful visualization technique
previously used in the time series data domain. This method facilitates
spotting interesting patterns across multi-sample data sets. By using the web
as our platform, we are exploiting its ubiquity to make exploring datasets
stored in different locations easier.  Computing only the necessary data points
for the visualization in the backend makes feasible exploring big data sets
across multiple samples.

Our library keeps the effective and elegant interface of the original Cubism,
allowing users to leverage its power for genomic data sets. By providing a
library, we maintain flexibility regarding how to use these resources. Users
can build full applications or use the library within their existing ones.

The companion lightweight server will facilitate the exploration of large
genomic datasets without affecting user experience by using indexed data sets.
Alternatively, users can create their own data sources to capture the details
of their environments� architecture.

To expose the power of DNAism, we accompany with DNAism a web
application\footnote{https://github.com/drio/bedbrowser} example that uses the
library and the bedserver backend to facilitate exploring and comparing genomic
metrics from different projects and samples.


\section*{Acknowledgement}

We want to thank Mike Bostock for his remarkable contributions to the
visualization field, beautifully captured on the release of D3~\citep{2011-d3}.

\paragraph{Funding\textcolon} NIH(NCRR) R24OD011173.

\bibliography{document}

\end{document}
