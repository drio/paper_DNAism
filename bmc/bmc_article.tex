%% BioMed_Central_Tex_Template_v1.06
%%                                      %
%  bmc_article.tex            ver: 1.06 %
%                                       %

%%IMPORTANT: do not delete the first line of this template
%%It must be present to enable the BMC Submission system to
%%recognise this template!!

%%%%%%%%%%%%%%%%%%%%%%%%%%%%%%%%%%%%%%%%%
%%                                     %%
%%  LaTeX template for BioMed Central  %%
%%     journal article submissions     %%
%%                                     %%
%%          <8 June 2012>              %%
%%                                     %%
%%                                     %%
%%%%%%%%%%%%%%%%%%%%%%%%%%%%%%%%%%%%%%%%%


%%%%%%%%%%%%%%%%%%%%%%%%%%%%%%%%%%%%%%%%%%%%%%%%%%%%%%%%%%%%%%%%%%%%%
%%                                                                 %%
%% For instructions on how to fill out this Tex template           %%
%% document please refer to Readme.html and the instructions for   %%
%% authors page on the biomed central website                      %%
%% http://www.biomedcentral.com/info/authors/                      %%
%%                                                                 %%
%% Please do not use \input{...} to include other tex files.       %%
%% Submit your LaTeX manuscript as one .tex document.              %%
%%                                                                 %%
%% All additional figures and files should be attached             %%
%% separately and not embedded in the \TeX\ document itself.       %%
%%                                                                 %%
%% BioMed Central currently use the MikTex distribution of         %%
%% TeX for Windows) of TeX and LaTeX.  This is available from      %%
%% http://www.miktex.org                                           %%
%%                                                                 %%
%%%%%%%%%%%%%%%%%%%%%%%%%%%%%%%%%%%%%%%%%%%%%%%%%%%%%%%%%%%%%%%%%%%%%

%%% additional documentclass options:
%  [doublespacing]
%  [linenumbers]   - put the line numbers on margins

%%% loading packages, author definitions

%\documentclass[twocolumn]{bmcart}% uncomment this for twocolumn layout and comment line below
\documentclass[twocolumn]{bmcart}

%%% Load packages
%\usepackage{amsthm,amsmath}
%\RequirePackage{natbib}
%\RequirePackage{hyperref}
%\usepackage[utf8]{inputenc} %unicode support
%\usepackage[applemac]{inputenc} %applemac support if unicode package fails
%\usepackage[latin1]{inputenc} %UNIX support if unicode package fails


%%%%%%%%%%%%%%%%%%%%%%%%%%%%%%%%%%%%%%%%%%%%%%%%%
%%                                             %%
%%  If you wish to display your graphics for   %%
%%  your own use using includegraphic or       %%
%%  includegraphics, then comment out the      %%
%%  following two lines of code.               %%
%%  NB: These line *must* be included when     %%
%%  submitting to BMC.                         %%
%%  All figure files must be submitted as      %%
%%  separate graphics through the BMC          %%
%%  submission process, not included in the    %%
%%  submitted article.                         %%
%%                                             %%
%%%%%%%%%%%%%%%%%%%%%%%%%%%%%%%%%%%%%%%%%%%%%%%%%


\def\includegraphic{}
\def\includegraphics{}


%%% Put your definitions there:
\startlocaldefs
\endlocaldefs


%%% Begin ...
\begin{document}

%%% Start of article front matter
\begin{frontmatter}

\begin{fmbox}
\dochead{Software}

\title{DNAism: Exploring genomic datasets on the web with Horizon Charts. }

%%%%%%%%%%%%%%%%%%%%%%%%%%%%%%%%%%%%%%%%%%%%%%
%%                                          %%
%% Enter the authors here                   %%
%%                                          %%
%% Specify information, if available,       %%
%% in the form:                             %%
%%   <key>={<id1>,<id2>}                    %%
%%   <key>=                                 %%
%% Comment or delete the keys which are     %%
%% not used. Repeat \author command as much %%
%% as required.                             %%
%%                                          %%
%%%%%%%%%%%%%%%%%%%%%%%%%%%%%%%%%%%%%%%%%%%%%%

\author[
   addressref={aff1},                   % id's of addresses, e.g. {aff1,aff2}
   corref={aff1},                       % id of corresponding address, if any
%   noteref={n1},                        % id's of article notes, if any
   email={deiros@bcm.edu}   % email address
]{\inits{DRD}\fnm{David} \snm{Rio Deiros}}
\author[
   addressref={aff1,aff2},
   email={deiros@bcm.edu}
]{\inits{RG}\fnm{Richard A.} \snm{Gibbs}}
\author[
   addressref={aff1,aff2},
   email={deiros@bcm.edu}
]{\inits{JR}\fnm{Jeffrey} \snm{Rogers}}


\address[id=aff1]{%                           % unique id
  \orgname{Human Genome Sequencing Center, Baylor College of Medicine}, % university, etc
  \street{One Baylor Plaza},                     %
  %\postcode{}                                % post or zip code
  \city{Houston},                              % city
  \cny{USA}                                    % country
}

\address[id=aff2]{%                           % unique id
  \orgname{Department of Molecular and Human Genetics, Baylor College of Medicine},
  \street{One Baylor Plaza},                     %
  %\postcode{}                                 % post or zip code
  \city{Houston},                              % city
  \cny{USA}                                    % country
}


%%%%%%%%%%%%%%%%%%%%%%%%%%%%%%%%%%%%%%%%%%%%%%
%%                                          %%
%% Enter short notes here                   %%
%%                                          %%
%% Short notes will be after addresses      %%
%% on first page.                           %%
%%                                          %%
%%%%%%%%%%%%%%%%%%%%%%%%%%%%%%%%%%%%%%%%%%%%%%

\begin{artnotes}
%\note{Sample of title note}     % note to the article
%\note[id=n1]{Equal contributor} % note, connected to author
\end{artnotes}


%%%%%%%%%%%%%%%%%%%%%%%%%%%%%%%%%%%%%%%%%%%%%%
%%                                          %%
%% The Abstract begins here                 %%
%%                                          %%
%% Please refer to the Instructions for     %%
%% authors on http://www.biomedcentral.com  %%
%% and include the section headings         %%
%% accordingly for your article type.       %%
%%                                          %%
%%%%%%%%%%%%%%%%%%%%%%%%%%%%%%%%%%%%%%%%%%%%%%

\begin{abstractbox}

\begin{abstract} % abstract

\parttitle{Background} %if any
Computational biologists daily face the need to explore massive amounts of
genomic data.  New visualization techniques help researchers navigate and
understand these big data. Horizon Charts are a relatively new visualization
method that, under the right circumstances, maximizes data density without
losing graphical perception. 


\parttitle{Results} %if any
Horizon Charts have been successfully applied to understand multi-metric time
series data. We have adapted an existing JavaScript library (Cubism) that
implements Horizon Charts for the time series domain so that it works
effectively with genomic datasets. We call this new library DNAism. 


\parttitle{Conclusions} %if any Users can use DNAism to leverate the power of
Horizon Charts to explore their own datasets. They can do so by creating their
own applications or expanding existing ones.

\end{abstract}



%%%%%%%%%%%%%%%%%%%%%%%%%%%%%%%%%%%%%%%%%%%%%%
%%                                          %%
%% The keywords begin here                  %%
%%                                          %%
%% Put each keyword in separate \kwd{}.     %%
%%                                          %%
%%%%%%%%%%%%%%%%%%%%%%%%%%%%%%%%%%%%%%%%%%%%%%

\begin{keyword}
\kwd{Bioformatics}
\kwd{Genomics}
\kwd{Sequencing}
\kwd{JavaScript}
\kwd{Web}
\kwd{Visualization}
\end{keyword}

% MSC classifications codes, if any
%\begin{keyword}[class=AMS]
%\kwd[Primary ]{}
%\kwd{}
%\kwd[; secondary ]{}
%\end{keyword}

\end{abstractbox}
\end{fmbox}% comment this for two column layout

\end{frontmatter}












%%%%%%%%%%%%%%%%%%%%%%%%%%%%%%%%%%%%%%%%%%%%%%
%%                                          %%
%% The Main Body begins here                %%
%%                                          %%
%% Please refer to the instructions for     %%
%% authors on:                              %%
%% http://www.biomedcentral.com/info/authors%%
%% and include the section headings         %%
%% accordingly for your article type.       %%
%%                                          %%
%% See the Results and Discussion section   %%
%% for details on how to create sub-sections%%
%%                                          %%
%% use \cite{...} to cite references        %%
%%  \cite{koon} and                         %%
%%  \cite{oreg,khar,zvai,xjon,schn,pond}    %%
%%  \nocite{smith,marg,hunn,advi,koha,mouse}%%
%%                                          %%
%%%%%%%%%%%%%%%%%%%%%%%%%%%%%%%%%%%%%%%%%%%%%%

%%%%%%%%%%%%%%%%%%%%%%%%% start of article main body
% <put your article body there>

%%%%%%%%%%%%%%%%
%% Background %%
%%
\section*{Background}

Sharing and communicating about large and intricate datasets produced by
Next-Gen sequencing can be a challenging task. Visual channels are an effective
way to explore data. However, the accelerating increase in data quantity is
pushing the limits of current approaches for representing these datasets
visually without sacrificing accuracy or graphical perception.  Data volume is
increasing vertically as throughput per sample increases, and horizontally as
studies involving large numbers of subjects become more feasible.  Thus, more
effective visualization techniques are needed to understand the most
challenging Next-Gen sequencing datasets.

Horizon Charts~\cite{time-in-the-horizon} have proved to be an effective~\cite{2009-horizon}
visualization approach when working with multi-metric time series encoded data.
On the other hand, BED\footnote{http://genome.ucsc.edu/FAQ/FAQformat.html}
files are the gold standard for capturing genomic metrics in the Next-Gen
sequencing domain. In time series, metrics are monitored over time, however,
BED files use genomic coordinates. We have adapted a time series JavaScript
library to the genomic domain. We call our new library DNAism.



\section*{Implementation}

Contrary to time series data, in genomic datasets, the variable under study is
associated with chromosomal coordinates instead of timestamps. We have modified
an existing time series data visualization library (based on D3~\cite{2011-d3})
called Cubism to support genome coordinate data. This makes DNAism
a flexible and effective tool to explore multi-sample genomic datasets using
Horizon Charts.

To visualize genomic datasets, we have modified most of the software components
of the original Cubism library.  The two major ones being 'context' and
'source'.  The 'context' component performs several functions, most
importantly, it defines the region of the genome we want to explore. This
component also specifies, in pixels, how much vertical space we have available
for the visualization.  The 'source' component parses the genomic raw data and
generates the data points necessary for visualization. Our library provides two
sources: 'bedfile' and 'bedserver'. Once the sources are created we can use the
metric component to instantiate metrics pointing to specific samples.  Finally,
the horizon component encapsulates the functionality necessary to create the
visual elements. Figure~\ref{fig:01} illustrates visualization of sample data.

One of the crucial features of DNAism is the ability to efficiently parse and
load the genomic data for visualization. We have provided two alternatives via
the bedfile and the bedserver sources. A bedfile is a simple solution that
loads all the genomic data in memory and returns the relevant data when
queried.  However, this approach is not adequate for larger datasets,
especially those involving multi-sample data. To handle such cases, the
bedserver source can be used. A bedserver is a dedicated server that
implements a RESTful API interface. The client's code running in the browser
can send queries to this server to obtain the data of interest.  The server
uses pre-indexed~\cite{tabix-li} data to speed up random access and returns
only the necessary information for the visualization back to the client.
Hence, this approach becomes much more scalable even with large sized genomic
data sets. We have implemented bedserver as a Python package.

Our source code has a decoupled interface that facilitates the extension of
this library to new data sources. DNAism is data agnostic. As a result, users
can create new sources to capture their specific backend peculiarities.


\section*{Results and Discussion}

% The Results and Discussion may be combined into a single section or presented
% separately. They may also be broken into subsections with short, informative
% headings. In any case what should be described is the functionality of the
% software together with data on how its performance and functionality compare
% with and improve on functionally similar existing software. There should then
% be a discussion of the intended use of the software, and the benefits that are
% envisioned together, if possible, with an outline for the planned future
% development of new features.




\section*{Conclusion}

% This should state clearly the main conclusions of the article and give a clear
% explanation of the importance and relevance of the software.

We introduce the genomics community to a powerful visualization technique
previously used in the time series data domain. This method facilitates
identification of abnormal patterns across multi-sample datasets.  In addition, this
approach helps to explore and visualize high density datasets more effectively,
thereby, helping the researchers to understand the data easily.

Our library keeps the effective and elegant interface of the original,
while allowing users to leverage its power for genomic data. By providing a
library, we maintain flexibility regarding how to use these resources. Users
can build full applications or use the library within their existing ones.

The companion lightweight server will facilitate the exploration of large
genomic datasets without affecting user experience by using indexed datasets.
Alternatively, users can create their own data sources to reflect the details
of their own environments.





\section*{Availability and requirements}

\begin{itemize}
  \item Project name: DNAism.
  \item Project home page: http://drio.github.io/dnaisml (main site), https://github.com/drio/dnaism (source code).
  \item Operating system(s): Platform Independent.
  \item Programming language: JavaScript.
  \item Other requirements: Modern Browser.
  \item License: Apache 2.0.
\end{itemize}



%%%%%%%%%%%%%%%%%%%%%%%%%%%%%%%%%%%
%%                               %%
%% Figures                       %%
%%                               %%
%% NB: this is for captions and  %%
%% Titles. All graphics must be  %%
%% submitted separately and NOT  %%
%% included in the Tex document  %%
%%                               %%
%%%%%%%%%%%%%%%%%%%%%%%%%%%%%%%%%%%

%%
%% Do not use \listoffigures as most will included as separate files

\section*{Figures}
\begin{figure}[h!]

\caption{
  \csentence{Horizon Charts} emerge from applying a set of changes to traditional line
  charts~\textbf{(A)}. We start by coloring the underlying area of the line chart,
  using different hues for positive and negative values. Next, we divide the
  graph in bands and apply a gradient of color that increases along with the
  quantitative value of the variable we are investigating~\textbf{(B)}. In the
  next step, negative values are flipped over the baseline~\textbf{(C)},
  effectively reducing the vertical space by two fold. In a final step, bands are
  collapsed making all of them start at the baseline and providing another level
  of space reduction~\textbf{(D)}.  
  \csentence{We used this technique} to rapidly identify problematic samples
  when performing quality control on large scale sequencing.
  You can see the read depth across whole genome sequences from 24 rhesus macaque
  samples (30x coverage) for genomic region Chr17:1.1M-1.2M~\textbf{(E)}.
  There are regions consistently underrepresented across all the
  samples and sample 32510 has low coverage across the whole genomic region.
  Note that the variable we are exploring in this example, read depth, does not
  contain negative values. Therefore, only green hues appear in~\textbf{(E)}.
}\label{fig:01}

\end{figure}





%%%%%%%%%%%%%%%%%%%%%%%%%%%%%%%%%%%%%%%%%%%%%%
%%                                          %%
%% Backmatter begins here                   %%
%%                                          %%
%%%%%%%%%%%%%%%%%%%%%%%%%%%%%%%%%%%%%%%%%%%%%%

\begin{backmatter}

\section*{Competing interests}

The authors declare that they have no competing interests.



\section*{Author's contributions}

DRD conceptualized, designed and developed the algorithms, workflow and
interface for the analysis and drafted the manuscript.  All authors contributed
in discussions and approved the final draft of the manuscript.



\section*{Acknowledgements}
  Text for this section \ldots

%%%%%%%%%%%%%%%%%%%%%%%%%%%%%%%%%%%%%%%%%%%%%%%%%%%%%%%%%%%%%
%%                  The Bibliography                       %%
%%                                                         %%
%%  Bmc_mathpys.bst  will be used to                       %%
%%  create a .BBL file for submission.                     %%
%%  After submission of the .TEX file,                     %%
%%  you will be prompted to submit your .BBL file.         %%
%%                                                         %%
%%                                                         %%
%%  Note that the displayed Bibliography will not          %%
%%  necessarily be rendered by Latex exactly as specified  %%
%%  in the online Instructions for Authors.                %%
%%                                                         %%
%%%%%%%%%%%%%%%%%%%%%%%%%%%%%%%%%%%%%%%%%%%%%%%%%%%%%%%%%%%%%

% if your bibliography is in bibtex format, use those commands:
\bibliographystyle{bmc-mathphys} % Style BST file
\bibliography{bmc_article}      % Bibliography file (usually '*.bib' )

% or include bibliography directly:
% \begin{thebibliography}
% \bibitem{b1}
% \end{thebibliography}


%%%%%%%%%%%%%%%%%%%%%%%%%%%%%%%%%%%
%%                               %%
%% Additional Files              %%
%%                               %%
%%%%%%%%%%%%%%%%%%%%%%%%%%%%%%%%%%%

\section*{Additional Files}
\subsection*{Additional file 1 --- The DNAism home page show an example how to create simple visualization step by step.}
\subsection*{Additional file 2 --- Source code, with comments, for the visualization in Figure 1.}


\end{backmatter}
\end{document}
